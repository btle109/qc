\documentclass{article}
\usepackage{graphicx} % Required for inserting images
\usepackage{tcolorbox}
\usepackage{amsthm}
\usepackage{amsmath}
\usepackage{amssymb}
\usepackage{enumitem}
\usepackage{hyperref}
\usepackage{braket}

\title{Quantum Computation Notes \\ \large Quantum Circuits}
\author{}
\date{}
\begin{document}
\maketitle
\vspace{-2cm}
\tableofcontents
\section{Single qubit operations}
Recall: a single qubit is $\ket{\psi} = \alpha \ket0 + \beta \ket1$ with $|\alpha|^2 + |\beta|^2 = 1$. Operations must preserve the norm, so we use unitary matrices for operations on one qubit. Useful gates include the Pauli matrices X,Y,Z,the Hadamard gate (H), phase gate (S), and the $\frac{\pi}{8}$ gate.
$$X \equiv \begin{bmatrix} 0 & 1 \\ 1 & 0\end{bmatrix}, Y \equiv \begin{bmatrix} 0 & -i \\ i & 0 \end{bmatrix}, Z \equiv \begin{bmatrix} 1 & 0 \\ 0 & -1\end{bmatrix}$$
$$H \equiv \frac{1}{\sqrt2}\begin{bmatrix} 1& 1 \\ 1 & -1\end{bmatrix}, S = \begin{bmatrix} 1 & 0 \\ 0 & i\end{bmatrix}, T = \begin{bmatrix} 1 & 0 \\ 0 & e^{i\frac{\pi}{4}}\end{bmatrix}$$
Then recall that we can express any qubit $\alpha \ket0 + \beta \ket1$ as $\alpha = \cos(\theta/2), b = e^{i\varphi} \sin(\theta/2) \iff \vec{v} = (\cos \varphi \sin \theta, \sin\varphi \sin \theta, \cos \theta)$ where $\vec{v}$ is the Bloch vector. 
\begin{tcolorbox}\textbf{Exercise 4.1:}
 Find the points on the Bloch sphere which correspond to the normalized eigenvectors of the different Pauli matrices.\\
The eigenvectors of the Pauli matrices correspond to the axes of the Bloch sphere.
	For example, we solve for $\ket{\psi}_{x+} = \frac{1}{\sqrt2}(\ket0 + \ket1)$, which for $\ket{\psi} = \cos(\theta/2)\ket0 + e^{i \phi} \sin(\theta/2)\ket{1}$
$$\implies cos (\frac{\theta}{2}) = \frac{1}{\sqrt{2}} \implies \theta = \frac{\pi}{2}$$
$$e^{i\phi} \sin(\theta/2) = \frac{1}{\sqrt{2}} \implies \phi = 0$$
$$\implies v = (\cos 0 \sin \frac{\pi}{2}, \sin 0 \sin \frac{\pi}{2}, \cos{\frac{\pi}{2}}) = (1,0,0)$$
\end{tcolorbox}
\newpage
The Pauli matrices allow us to define the very useful rotation operators:
$$R_x(\theta) = e^{-i\theta X/2} = \cos \frac{\theta}{2}I -  i\sin \frac{\theta}{2}X = \begin{bmatrix} \cos \frac{\theta}{2} & -i \sin \frac{\theta}{2} \\ -i \sin \frac{\theta}{2} & \cos \frac{\theta}{2} \end{bmatrix}$$
$$R_y(\theta) = e^{-i\theta Y/2} = \cos \frac{\theta}{2}I -  i\sin \frac{\theta}{2}Y = \begin{bmatrix} \cos \frac{\theta}{2} & -\sin \frac{\theta}{2} \\ \sin \frac{\theta}{2} & \cos \frac{\theta}{2} \end{bmatrix}$$
$$R_z(\theta) = e^{-i\theta Z/2} = \cos \frac{\theta}{2}I -  i\sin \frac{\theta}{2}Z = \begin{bmatrix} e^{-i\theta/2} & 0 \\ 0 & e^{i\theta/2} \end{bmatrix}$$
From here, we can define a rotation around $\hat{n} = (n_x, n_y, n_z)$, where Pauli rotations are a case of rotating around a Pauli eigenvector as $\hat{n}$ or conversely we can view the general rotation as some of each Pauli rotation.
\begin{tcolorbox}\textbf{Exercise 4.2}
 Let $x$ be a real number and $A$ a matrix such that $A^2 = I$. Show that $exp(iAx) = \cos(x)I + i \sin(x)A$\\

	$$exp(iAx) = 1 + iAx  + \frac{1}{2!}(iAx)^2 + \frac{1}{3!}(iAx)^3 + \frac{1}{4!}(iAx)^4 + \dots$$
	$$=  1 + iAx - \frac{1}{2!}x^2 - \frac{1}{3!}iAx^3 + \frac{1}{4!}x^4+\dots$$
	$$= 1 - \frac{1}{2!}x^2 + \frac{1}{4!}x^4 + \dots + iAx - \frac{1}{3!}iAx^3 + \dots$$
	$$= \cos(x)I + i\sin(x)A$$
\end{tcolorbox}

\begin{tcolorbox}\textbf{Exercise 4.3:}
	Show that, up to a global phase, the $\frac{\pi}{8}$ gate satisfies $T = R_z(\frac{\pi}{4})$.\\

	\begin{align*}
		R_z(\frac{\pi}{4}) &= \begin{bmatrix}e^{-i\frac{\pi}{8}} & 0 \\ 0 & e^{i \frac{\pi}{8}}\end{bmatrix}\\
				  &\equiv e^{i\frac{\pi}{8}} \begin{bmatrix}e^{-i\frac{\pi}{8}} & 0 \\ 0 & e^{i \frac{\pi}{8}}\end{bmatrix}\\
				  &= \begin{bmatrix} 1 & 0 \\ 0 & e^{i \frac{\pi}{4}}\end{bmatrix} = T
\end{align*}
\end{tcolorbox}

\begin{tcolorbox}\textbf{Exercise 4.4:}
Express the Hadamard gate $H$ as a product of $R_x$ and $R_z$ rotations and $e^{i\phi}$ for some $\phi$.\\
	We see that $R_x(\frac{\pi}{2}]) = \frac{1}{\sqrt{2}}\begin{bmatrix} 1 & -i \\ -i & 1\end{bmatrix}$, which when multiplied on the left and right by $\begin{bmatrix} 1 & 0 \\ 0 & i\end{bmatrix}$ is the Hadamard gate. We then can see that $R_z(\frac{\pi}{2})  = \begin{bmatrix} e^{-i\frac{\pi}{4}} & 0 \\ 0 & e^{i\frac{\theta}{2}}\end{bmatrix} \implies e^{i\frac{\pi}{4}} R_z(\frac{\pi}{4}) = \begin{bmatrix} 1 & 0 \\ 0 & i\end{bmatrix} $  
	$$H = e^{i\frac{\pi}{2}}R_z(\frac{\pi}{2})R_x(\frac{\pi}{2}) R_z(\frac{\pi}{2})$$
\end{tcolorbox}

\begin{tcolorbox}\textbf{Exercise 4.5:} 
 Prove that $(\hat{n} \cdot \sigma)^2 = I$ and use this to verify Equation (4.8).\\
	$$(\hat{n} \cdot \sigma)^2 = (\begin{bmatrix}v_3 & v_1 - iv_2 \\ v_1 + iv_2 & -v_3 \end{bmatrix})^{2}$$
	$$= \begin{bmatrix} ||v||^2 & 0 \\ 0 & ||v||^2 \end{bmatrix}= I$$
	Then, (4.8) follows immediately from the exponential form of a complex number.
\end{tcolorbox}
\begin{tcolorbox}\textbf{Exercise 4.7:}
	Show that $XYX = -Y$ and use this to prove $XR_y(\theta)X = R_y(-\theta)$.
	$$XYX = \begin{bmatrix} 0 & 1 \\ 1 & 0\end{bmatrix} \begin{bmatrix} 0 & -i \\ i & 0 \end{bmatrix} \begin{bmatrix} 0 & 1 \\ 1 & 0\end{bmatrix} = \begin{bmatrix} 0 & 1 \\ 1 & 0\end{bmatrix}\begin{bmatrix} -i& 0 \\ 0 & i \end{bmatrix}=\begin{bmatrix}0 & i \\ -i & 0\end{bmatrix} = -Y$$
	We also can use prop. of X,Y anticommuting.
	We know$XY=-YX \iff XYX = -YXX = -Y$.
	Let $x =\frac{\theta}{2}$. As we see, multiplying left and right by $X$ swaps the rows, then the columns.
	$$\iff XR_y(\theta)X = \begin{bmatrix} \cos x & \sin x \\ -\sin x & \cos x\end{bmatrix} = \cos x I + i \sin x Y = R_y(-\theta)$$
\end{tcolorbox}
\begin{tcolorbox}\textbf{Exercise 4.8:} An arbitrary single qubit unitary operator can be written in the form
	$$U = \text{exp}(i\alpha)R_{\hat{n}}(\theta)$$
	for some real numbers $\alpha, \theta$ and for a real three dimensional unit vector $\hat{n}$.\\
	First, we show that any unitary matrix can be expressed as $U = e^{iH}$ where $H$ is a Hermitian matrix. We see that $(e^{iH})^\dag = e^{-iH} \implies UU^\dag = e^{iH - iH} =e^{0_n} = I_n$. Then, see that we can express any $H$ as $H = \begin{bmatrix} a & c-di \\c+di & b \end{bmatrix}$. Matching with $\vec{v} \cdot \sigma$, we can see that we can match $v_1 = c, d = v_2$. To allow for $a \neq -b$, we add constant $\alpha$ to both, giving the system of equations: 
	$$ v_3 + \alpha = a, -v_3 + \alpha = b \iff v_3 = \alpha - b, v_3 = a- \alpha$$
	$$\iff \alpha = \frac{a+b}{2}$$.
	Thus we have $v = (c,d, a-\alpha)$ and we can express $H = \alpha I + \vec{v} \cdot \sigma$. Comparing to the $R_{\hat{n}}(\theta)$ rotation, we see that $-\theta \hat{n}/2 = \vec{v}$ and using the fact that $\hat{n} = \frac{\vec{v}}{|v|}$:
	$$\iff \vec{v} = -\frac{\vec{v}\theta}{2|\vec{v}|} \iff \theta = -2|\vec{v}| \iff \vec{v} = -\frac{\theta\vec{n}}{2} $$
	Substituting into our expression for $H = \alpha I + \vec{v} \cdot \sigma = \alpha I - \frac{\theta\vec{n}}{2}$
	$$\iff U = e^{iH} = e^{i(\alpha I - \frac{\theta \vec{n \cdot \sigma}}{2})} = e^{i\alpha} e^{-i\frac{\theta n \cdot \sigma}{2}} = e^{i \alpha} R_{\hat{n}}{(\theta)}$$
	We showed previously that the phase gate $\begin{bmatrix} 1 & 0 \\ 0 & i\end{bmatrix} = e^{i\frac{\pi}{4}}R_z(\frac{\pi}{4})$.\\
For the Hadamard gate, our intuition shows us that the H gate is a rotation around vector $\hat{n} = (\frac{1}{\sqrt{2}}, 0 , \frac{1}{\sqrt{2}})$ of magnitude $\frac{\pi}{2}$. 
\end{tcolorbox}
Theorem 4.1 (ZY Decomposition for a single qubit) Suppose $U$ is a unitary operation on one qubit. Then, we can find $\alpha, \beta, \gamma, \delta$ such that:
$$U = e^{i\alpha} R_{z}(\beta) R_y(\gamma) R_z(\delta)$$.
$$U = \begin{bmatrix} e^{i(\alpha - \beta/2 - \delta/2)} \cos \frac{\gamma}{2} & -e^{i(\alpha - \beta/2 - \delta/2)} \sin \frac{\gamma}{2} \\ e^{\alpha + \beta/2 + \delta/2} \sin \frac{\gamma}{2}  & e^{i(\alpha + \beta/2 + \delta/2)} \cos \frac{\gamma}{2} \end{bmatrix}$$
\begin{tcolorbox}\textbf{Exercise 4.9} Explain why any single qubit unitary may be written in the form in the theorem.\\
	Suppose we have arbitrary unitary $U = \begin{bmatrix} a & b \\ c & d\end{bmatrix}$. We see that $U^\dag U = I$ and $UU^\dag = I$ imply that $|a|^2 + |c|^2 = 1, |b|^2 +|d|^2 = 1, |a|^2 + |b|^2 = 1, |b|^2 + |d|^2 =1$.\\
	We can satisfy this when $a \equiv d,b \equiv c$. For some complex number $z = re^{i\theta}$, we see that $|z_1|^2 + |z_2|^2 =1 \iff r_1^2 + r^2_2 = 1 \iff r_1 = \cos\theta, r_2 = \sin \theta$. Thus we see that $a = e^{i\theta_a} \cos \theta, b =e^{i\theta_b} \sin \theta$.\\
	Thus we have $$ U = \begin{bmatrix} e^{i\theta_a} \cos\theta & e^{i \theta_b} \sin \theta \\ e^{i\theta_c} \sin\theta &e^{i\theta_d} \cos\theta \end{bmatrix}$$. Since the columns of a unitary matrix are orthonormal, it must be that $$e^{i(\theta_a - \theta_b)} \cos\theta \sin\theta + e^{i (\theta_c - \theta_d)} \cos\theta \sin\theta = 0 \iff e^{i(\theta_a - \theta_b)} + e^{i(\theta_c - \theta_d)}= 0$$
	$$\iff \theta_a - \theta_b = \theta_c - \theta_d + \pi$$
	Choose $\theta_a = \alpha - \frac{\beta}{2} - \frac{\delta}{2}, \theta_c = \alpha + \frac{\beta}{2} - \frac{\delta}{2}, \theta_d = \alpha + \frac{\beta}{2} + \frac{\delta}{2}$. Choose $\theta = \frac{\gamma}{2}$.
	We can solve for $\theta_b$ and see that:
	$$\alpha - \frac{\beta}{2} - \frac{\delta}{2} - \theta_c + \theta_d - \pi = \theta_b$$
	$$\iff \theta_b = \alpha - \frac{\beta}{2} + \frac{\delta}{2} - \pi$$
	$$\iff e^{i\theta_b} = e^{-i\pi} e^{i(\alpha - \frac{\beta}{2} + \frac{\delta}{2})} = - e^{i(\alpha - \frac{\beta}{2} + \frac{\delta}{2})} $$
	Thus we see $U$ can be represented as in the theorem.

\end{tcolorbox}
\begin{tcolorbox}\textbf{Exercise 4.10} (X-Y Decomposition of Rotations) Give a decomposition analogous to Theorem 4.1 but using $R_x$ instead of $R_z$. \\
	We can write any arbitrary unitary $U$ as $U= HVH^\dag$ (define $V = H^\dag UH$). Then, decompose $V$ using $ZY$ decomposition. 
	$$U = HVH = e^{i\alpha} H(R_z(\beta)R_y(\gamma)R_z(\delta))H$$
	$$= e^{i\alpha} HR_z(\beta)H HR_y(\gamma)H HR_z(\delta)H$$
	Then, we see that $HR_z(\beta)H = R_x(\beta), HR_y (\gamma)H = R_y(-\gamma)$, which is easy to verify from the Euler decomposition of complex number.
	$$= e^{i\alpha} R_x(\beta) R_y(-\gamma) R_x(\delta)$$
\end{tcolorbox}
\begin{tcolorbox}\textbf{Exercise 4.11} Suppose $\hat{m}, \hat{n}$ are non-parallel real unit vectors in three dimensions. Use Thm. 4.1 to show an arbitrary single qubit $U$ may be written as $U = e^{i\alpha} R_{\hat{n}}(\beta) R_{\hat{m}} R_{\hat{n}}(\delta)$\\
	Consider unitary operator $V$ such that $VZV^\dag = \hat{n} \cdot \sigma$.We can represent any unitary $U$ as $U = VWV^\dag$ for unitary $W = V^\dag U V$. Using Thm 4.1 on W,  we see that
	$$U = e^{i\alpha}V(R_z(\beta) R_y (\gamma) R_z(\delta))V = e^{i\alpha} R_{\hat{n}}(\theta') VR_y (\phi')V^\dag R_{\hat{n}} (\tau')$$
	Suppose $VyV^\dag = \hat{k}$. Then, we see that since $y$ is perpendicular to $z$, the unitary transformation $V$ (preserves angles), it must be that $k$ is perpendicular to $\hat{n}$. Since $\hat{m}$ is not parallel to $\hat{n}$, it can be decomposed into a component parallel and a component perpendicular (along $k$).
	$$= e^{i\alpha} R_{\hat{n}}(\theta) R_{\hat{m}} (\phi) R_{\hat{n}}(\tau)$$
\end{tcolorbox}
Corollary 4.2: Suppose $U$ is a unitary gate on a single qubit. Then, there exists unitary operators $A,B,C$ on a single qubit such that $ABC = I$, $U = e^{o\alpha} AXBXC$ where $\alpha$ is some overall phase factor.\\
Set $A \equiv R_z(\beta)R_y(\gamma/2), B\equiv R_y(-\gamma/2)R_z(-\frac{\delta+\beta}{2}), C \equiv R_z(\frac{\delta-\beta}{2})$ for $ZY$ decomposition of $U = R_z(\beta) R_y(\gamma) R_z(\delta)$
\begin{tcolorbox}\textbf{Exercise 4.12} Give A,B,C and $\alpha$ for the Hadamard gate.\\
	We can find from the $ZY$ decomposition of $H$ that $U = e^{i\frac{\pi}{2}} R_z(\pi) R_y(\frac{\pi}{2})$. These values can be easily derived by solving a system of equation for $\alpha, \beta, \delta$ in the $ZY$ decomposition. \\
	Thus we have $\alpha = \frac{\pi}{2}, \beta = \pi, \delta =0, \gamma = \frac{\pi}{2}$
	$$\iff A= R_z(\pi)R_y(\frac{\pi}{4}), B = R_y(-\frac{\pi}{4})R_z(\frac{\pi}{2}), C = R_z(\frac{\pi}{2})$$
\end{tcolorbox}
\begin{tcolorbox}\textbf{Exercise 4.13 (Circuit identities)} Prove the identites:
	$$HXH = Z, HYH = -Y, HZH = X$$
The first two are easily shown by matrix multiplication.\\
The third can beshown using the first. We see that $HXH = Z \iff HHXHH = HZH \iff X = HZH$.
\end{tcolorbox}
\begin{tcolorbox}\textbf{Exercise 4.14} Show $HTH = R_x(\frac{\pi}{4})$\\
	We have previously shown that $T = R_z(\frac{\pi}{4})$.
	$$HZH = X \iff HR_z(\frac{\pi}{4})H = R_x(\frac{\pi}{4})$$
\end{tcolorbox}

\begin{tcolorbox}\textbf{Exercise 4.15 (Composition of single qubit operations)} \begin{enumerate}
	\item Prove that if a rotation through an angle $\beta_1$ about axis $\hat{n_1}$ is followed by a rotation $\beta_2$ through $\hat{n}_2$ then the overall rotation through an angle $\beta_{12}$ about axis $\hat{n}_{12}$ given by:
		$$c_{12} = c_1 c_2 - s_1 s_2 \hat{n_1} \cdot \hat{n_2}$$
	$$s_{12} \hat{n}_{12} = s_1c_2 \hat{n_1} + c_1 s_2 \hat{n_2} - s_1 s_2 \hat{n_2} \times \hat{n_1}$$
	where $c_i = \cos (\beta_i/2), s_i = \sin(\beta_i/2), c_{12} = \cos(\beta_{12}/2), s_{12} = \sin(\beta_{12}/2)$.\\
	We see that $R_{\hat{n_i}}(\beta_i) = c_i I + i s_i (\hat{n_1} \cdot \sigma)$
	$$\iff R_{\hat{n_2}}(\beta_2) R_{\hat{n_1}}(\beta_1) = (c_2 I + is_2(\hat{n_2} \cdot \sigma))(c_1 I + is_1(\hat{n_1} \cdot \sigma))$$
	$$ = c_1c_2 I + i s_1c_2 (\hat{n_1} \cdot \sigma) + i c_1s_2 (\hat{n_2} \cdot \sigma) - s_1s_2 (\hat{n_1} \cdot \sigma)(\hat{n_2} \cdot \sigma)$$
	Using property that $(a \cdot \sigma)(b\cdot \sigma) = (a \cdot b)I + i (a \times b)$,
	$$ = c_1c_2 I = is_1c_2 (\hat{n_1} \cdot \sigma) + i c_1s_2 (\hat{n_2} \cdot \sigma) - s_1s_2 ((n_2 \cdot n_1)I + i(n_2 \times n_1) \cdot \sigma)$$
	Grouping real/imaginary parts, we see
	$$(c_1c_2 - s_1s_2 (\hat{n_2} \cdot \hat{n_1})) + i(s_1c_2 \hat{n_1} + s_2c_1 \hat{n_2} - s_1s_2 (n_2 \times n_1)) \cdot \sigma$$
\item It is clear to see that $\beta_1 = \beta_2$ and $\hat{n_1} = \hat{z}$ that the equations simplify to what is in the textbook.
\end{enumerate}
\end{tcolorbox}

\end{document}

