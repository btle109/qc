\documentclass{article}
\usepackage{graphicx} % Required for inserting images
\usepackage{tcolorbox}
\usepackage{amsthm}
\usepackage{amsmath}
\usepackage{amssymb}
\usepackage{enumitem}
\usepackage{hyperref}
\usepackage{braket}

\title{Quantum Computation Notes \\ \large Quantum Circuits}
\author{}
\date{}
\begin{document}
\maketitle
\vspace{-2cm}
\tableofcontents
\section{Single qubit operations}
Recall: a single qubit is $\ket{\psi} = \alpha \ket0 + \beta \ket1$ with $|\alpha|^2 + |\beta|^2 = 1$. Operations must preserve the norm, so we use unitary matrices for operations on one qubit. Useful gates include the Pauli matrices X,Y,Z,the Hadamard gate (H), phase gate (S), and the $\frac{\pi}{8}$ gate.
$$X \equiv \begin{bmatrix} 0 & 1 \\ 1 & 0\end{bmatrix}, Y \equiv \begin{bmatrix} 0 & -i \\ i & 0 \end{bmatrix}, Z \equiv \begin{bmatrix} 1 & 0 \\ 0 & -1\end{bmatrix}$$
$$H \equiv \frac{1}{\sqrt2}\begin{bmatrix} 1& 1 \\ 1 & -1\end{bmatrix}, S = \begin{bmatrix} 1 & 0 \\ 0 & i\end{bmatrix}, T = \begin{bmatrix} 1 & 0 \\ 0 & e^{i\frac{\pi}{4}}\end{bmatrix}$$
Then recall that we can express any qubit $\alpha \ket0 + \beta \ket1$ as $\alpha = \cos(\theta/2), b = e^{i\varphi} \sin(\theta/2) \iff \vec{v} = (\cos \varphi \sin \theta, \sin\varphi \sin \theta, \cos \theta)$ where $\vec{v}$ is the Bloch vector. 
\begin{tcolorbox}\textbf{Exercise 4.1:}
 Find the points on the Bloch sphere which correspond to the normalized eigenvectors of the different Pauli matrices.\\
The eigenvectors of the Pauli matrices correspond to the axes of the Bloch sphere.
	For example, we solve for $\ket{\psi}_{x+} = \frac{1}{\sqrt2}(\ket0 + \ket1)$, which for $\ket{\psi} = \cos(\theta/2)\ket0 + e^{i \phi} \sin(\theta/2)\ket{1}$
$$\implies cos (\frac{\theta}{2}) = \frac{1}{\sqrt{2}} \implies \theta = \frac{\pi}{2}$$
$$e^{i\phi} \sin(\theta/2) = \frac{1}{\sqrt{2}} \implies \phi = 0$$
$$\implies v = (\cos 0 \sin \frac{\pi}{2}, \sin 0 \sin \frac{\pi}{2}, \cos{\frac{\pi}{2}}) = (1,0,0)$$
\end{tcolorbox}
\newpage
The Pauli matrices allow us to define the very useful rotation operators:
$$R_x(\theta) = e^{-i\theta X/2} = \cos \frac{\theta}{2}I -  i\sin \frac{\theta}{2}X = \begin{bmatrix} \cos \frac{\theta}{2} & -i \sin \frac{\theta}{2} \\ -i \sin \frac{\theta}{2} & \cos \frac{\theta}{2} \end{bmatrix}$$
$$R_y(\theta) = e^{-i\theta Y/2} = \cos \frac{\theta}{2}I -  i\sin \frac{\theta}{2}Y = \begin{bmatrix} \cos \frac{\theta}{2} & -\sin \frac{\theta}{2} \\ \sin \frac{\theta}{2} & \cos \frac{\theta}{2} \end{bmatrix}$$
$$R_z(\theta) = e^{-i\theta Z/2} = \cos \frac{\theta}{2}I -  i\sin \frac{\theta}{2}Z = \begin{bmatrix} e^{-i\theta/2} & 0 \\ 0 & e^{i\theta/2} \end{bmatrix}$$
From here, we can define a rotation around $\hat{n} = (n_x, n_y, n_z)$, where Pauli rotations are a case of rotating around a Pauli eigenvector as $\hat{n}$ or conversely we can view the general rotation as some of each Pauli rotation.
\begin{tcolorbox}\textbf{Exercise 4.2}
 Let $x$ be a real number and $A$ a matrix such that $A^2 = I$. Show that $exp(iAx) = \cos(x)I + i \sin(x)A$\\

	$$exp(iAx) = 1 + iAx  + \frac{1}{2!}(iAx)^2 + \frac{1}{3!}(iAx)^3 + \frac{1}{4!}(iAx)^4 + \dots$$
	$$=  1 + iAx - \frac{1}{2!}x^2 - \frac{1}{3!}iAx^3 + \frac{1}{4!}x^4+\dots$$
	$$= 1 - \frac{1}{2!}x^2 + \frac{1}{4!}x^4 + \dots + iAx - \frac{1}{3!}iAx^3 + \dots$$
	$$= \cos(x)I + i\sin(x)A$$
\end{tcolorbox}

\begin{tcolorbox}\textbf{Exercise 4.3:}
	Show that, up to a global phase, the $\frac{\pi}{8}$ gate satisfies $T = R_z(\frac{\pi}{4})$.\\

	\begin{align*}
		R_z(\frac{\pi}{4}) &= \begin{bmatrix}e^{-i\frac{\pi}{8}} & 0 \\ 0 & e^{i \frac{\pi}{8}}\end{bmatrix}\\
				  &\equiv e^{i\frac{\pi}{8}} \begin{bmatrix}e^{-i\frac{\pi}{8}} & 0 \\ 0 & e^{i \frac{\pi}{8}}\end{bmatrix}\\
				  &= \begin{bmatrix} 1 & 0 \\ 0 & e^{i \frac{\pi}{4}}\end{bmatrix} = T
\end{align*}
\end{tcolorbox}

\begin{tcolorbox}\textbf{Exercise 4.4:}
Express the Hadamard gate $H$ as a product of $R_x$ and $R_z$ rotations and $e^{i\phi}$ for some $\phi$.\\
	We see that $R_x(\frac{\pi}{2}]) = \frac{1}{\sqrt{2}}\begin{bmatrix} 1 & -i \\ -i & 1\end{bmatrix}$, which when multiplied on the left and right by $\begin{bmatrix} 1 & 0 \\ 0 & i\end{bmatrix}$ is the Hadamard gate. We then can see that $R_z(\frac{\pi}{2})  = \begin{bmatrix} e^{-i\frac{\pi}{4}} & 0 \\ 0 & e^{i\frac{\theta}{2}}\end{bmatrix} \implies e^{i\frac{\pi}{4}} R_z(\frac{\pi}{4}) = \begin{bmatrix} 1 & 0 \\ 0 & i\end{bmatrix} $  
	$$H = e^{i\frac{\pi}{2}}R_z(\frac{\pi}{2})R_x(\frac{\pi}{2}) R_z(\frac{\pi}{2})$$
\end{tcolorbox}

\begin{tcolorbox}\textbf{Exercise 4.5:} 
 Prove that $(\hat{n} \cdot \sigma)^2 = I$ and use this to verify Equation (4.8).\\
	$$(\hat{n} \cdot \sigma)^2 = (\begin{bmatrix}v_3 & v_1 - iv_2 \\ v_1 + iv_2 & -v_3 \end{bmatrix})^{2}$$
	$$= \begin{bmatrix} ||v||^2 & 0 \\ 0 & ||v||^2 \end{bmatrix}= I$$
	Then, (4.8) follows immediately from the exponential form of a complex number.
\end{tcolorbox}
\begin{tcolorbox}\textbf{Exercise 4.7:}
	Show that $XYX = -Y$ and use this to prove $XR_y(\theta)X = R_y(-\theta)$.
	$$XYX = \begin{bmatrix} 0 & 1 \\ 1 & 0\end{bmatrix} \begin{bmatrix} 0 & -i \\ i & 0 \end{bmatrix} \begin{bmatrix} 0 & 1 \\ 1 & 0\end{bmatrix} = \begin{bmatrix} 0 & 1 \\ 1 & 0\end{bmatrix}\begin{bmatrix} -i& 0 \\ 0 & i \end{bmatrix}=\begin{bmatrix}0 & i \\ -i & 0\end{bmatrix} = -Y$$
	Let $x =\frac{\theta}{2}$. As we see, multiplying left and right by $X$ swaps the rows, then the columns.
	$$\iff XR_y(\theta)X = \begin{bmatrix} \cos x & \sin x \\ -\sin x & \cos x\end{bmatrix} = \cos x I + i \sin x Y = R_y(\theta)$$
\end{tcolorbox}
\begin{tcolorbox}\textbf{Exercise 4.8:} An arbitrary single qubit unitary operator can be written in the form
	$$U = \text{exp}(i\alpha)R_{\hat{n}}(\theta)$$
	for some real numbers $\alpha, \theta$ and for a real three dimensional unit vector $\hat{n}$.\\
	First, we show that any unitary matrix can be expressed as $U = e^{iH}$ where $H$ is a Hermitian matrix. We see that $(e^{iH})^\dag = e^{-iH} \implies UU^\dag = e^{iH - iH} =e^{0_n} = I_n$. Then, see that we can express any $H$ as $H = \begin{bmatrix} a & c-di \\c+di & b \end{bmatrix}$. Matching with $\vec{v} \cdot \sigma$, we can see that we can match $v_1 = c, d = v_2$. To allow for $a \neq -b$, we add constant $\alpha$ to both, giving the system of equations: 
	$$ v_3 + \alpha = a, -v_3 + \alpha = b \iff v_3 = \alpha - b, v_3 = a- \alpha$$
	$$\iff \alpha = \frac{a+b}{2}$$.
	Thus we have $v = (c,d, a-\alpha)$ and we can express $H = \alpha I + \vec{v} \cdot \sigma$. Comparing to the $R_{\hat{n}}(\theta)$ rotation, we see that $-\theta \hat{n}/2 = \vec{v}$ and using the fact that $\hat{n} = \frac{\vec{v}}{|v|}$:
	$$\iff \vec{v} = -\frac{\vec{v}\theta}{2|\vec{v}|} \iff \theta = -2|\vec{v}| \iff \vec{v} = -\frac{\theta\vec{n}}{2} $$
	Substituting into our expression for $H = \alpha I + \vec{v} \cdot \sigma = \alpha I - \frac{\theta\vec{n}}{2}$
	$$\iff U = e^{iH} = e^{i(\alpha I - \frac{\theta \vec{n \cdot \sigma}}{2})} = e^{i\alpha} e^{-i\frac{\theta n \cdot \sigma}{2}} = e^{i \alpha} R_{\hat{n}}{(\theta)}$$
	We showed previously that the phase gate $\begin{bmatrix} 1 & 0 \\ 0 & i\end{bmatrix} = e^{i\frac{\pi}{4}}R_z(\frac{\pi}{4})$ and the equivalence regarding the Hadamard gate.
\end{tcolorbox}
\end{document}
