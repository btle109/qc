\documentclass{article}
\usepackage{graphicx} % Required for inserting images
\usepackage{tcolorbox}
\usepackage{amsthm}
\usepackage{amsmath}
\usepackage{amssymb}
\usepackage{enumitem}
\usepackage{hyperref}
\usepackage{braket}
\title{Exercises from Nielsen and Chuang}
\author{}
\date{}

\begin{document}
\maketitle
\tableofcontents
\newpage
\section{2.1.5 Eigenvectors and eigenvalues}
(2.11) Find eigenvectors, eigenvalues, and diagonal representations of Pauli X,Y,Z.
\begin{tcolorbox}\textbf{Solution:}
	The eigenvalues of all Pauli matrices are $1,-1$.
	The eigenvectors are as follows:
	$$ \ket{\psi}_{x+} = \frac{1}{\sqrt2}(\ket0 + \ket1), \ket{\psi}_{x-}=\frac{1}{\sqrt2}(\ket0 - \ket1)$$
	$$\ket{\psi}_{y+} = \frac{1}{\sqrt2}(\ket0 + i\ket1), \ket{\psi}_{y-} = \frac{1}{\sqrt2}(\ket0 -i\ket1)$$
	$$\ket{\psi}_{z+} = \ket0, \ket{\psi}_{z-} = \ket{1}$$
	A diagonal representation is defined as $A = \sum_i \lambda_i \bra{i}\ket{i}$ where the vectors $i$ form an orthonormal set of eigenvectors. Thus, the diagonal representations of any Pauli matrix is:
	$$\sigma = \ket{\psi}_{+}\bra{\psi}_{+} - \ket{\psi}_{-}\bra{\psi}_{-}$$
\end{tcolorbox}
(2.12) Prove the matrix $\begin{bmatrix} 1 & 0 \\ 1 & 1 \end{bmatrix}$ is not diagonalizable.
\begin{tcolorbox}\textbf{Solution:}
	The eigenvalues of the matrix are $1$ with algebraic multiplicity $2$ (easily seen since it is a lower triangular matrix). Then, the eigenvectors are $\begin{bmatrix} 0 \\ \alpha \end{bmatrix}$, for $\alpha \in \mathbb{C}$.

A matrix is diagonalizable iff there exists a basis of eigenvectors of the matrix. There cannot be a matrix of eigenvectors since there is only one linearly independent eigenvector of the matrix, which is less than the dimension of the matrix. Thus it cannot form a basis and it cannot be diagonalizable.
\end{tcolorbox}
\section{2.1.6 Adjoints and Hermitian Operators}


\section{2.1.7 Tensor Products}
dog
\section{2.2.2 Evolution}
dog
\section{2.2.3 Quantum Measurement}
dog
\section{2.2.6 POVM Measurements}
dog
\section{2.2.8 Composite Systems}
dog
\section{2.3 Superdense Coding}
dog
\section{2.4.2 Density Operator Properties}
dog
\section{2.4.3 Reduced Density Operator}
dog

\end{document}
