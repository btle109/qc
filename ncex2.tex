\documentclass{article}
\usepackage{graphicx} % Required for inserting images
\usepackage{tcolorbox}
\usepackage{amsthm}
\usepackage{amsmath}
\usepackage{amssymb}
\usepackage{enumitem}
\usepackage{hyperref}
\usepackage{braket}
\title{Exercises from Nielsen and Chuang, Part II}
\author{}
\date{}

\begin{document}
\maketitle
\tableofcontents
\newpage
\section{2.18 Operator Functions}
(2.35) Let $v$ be any real, three-dimensional unit vector and $\theta$ a real number. Prove that $exp(i\theta v\cdot \sigma) = \cos(\theta)I + i \sin(\theta)v\cdot \sigma$.
\begin{tcolorbox}\textbf{Solution:}
	$$exp(i\theta v\cdot \sigma) = 1 + i\theta v\sigma - \frac{1}{2!}(\theta^2) - \frac{1}{3!}(i\theta v\sigma)+ \dots$$
	$$= \cos(\theta)I + i \sin(\theta)v \cdot \sigma$$
\end{tcolorbox}
(2.36) Show the Pauli matrices except for $I$ have trace zero.
\begin{tcolorbox}\textbf{Solution:}
	Trivial.
\end{tcolorbox}
(2.37-2.38) Prove the trace is cylic and linear.
\begin{tcolorbox}
	Expand matrix multiplication, sum diagonals. 
\end{tcolorbox}
(2.39) Show that $(A,B) \equiv tr(A^\dag B)$ is an inner product, show that $L_V$ has dimensions $d^2$ if $V$ has d dimension, and find an orthonormal basis on Hermitian matrices for Hilbert space $L_V$.
\section{2.2.3 Quantum Measurement}
(2.57) Suppose $\{L_l\}$ and $\{ M_m\}$ are two sets of measurement operators. Show that a measurement is defined by the measurement operators $\{ L_l\}$ followed by a measurement defined by the measurement operators $\{ M_m\}$ is physically equivalent to a single measurement defined by measurement operators $\{N_{lm}\}$ where $N_{lm} \equiv M_m L_l$
\begin{tcolorbox}\textbf{Solution:}
	Suppose we have vector $\ket{\psi}$ and the measurements defined above. We want to show that performing $L_l$ and $M_m$ is equivalent to performing measurement $N_{lm} \equiv M_m L_l$. Then, perform measurement $L_l$ on $\ket\psi \to \frac{L_l\ket{\psi}}{\sqrt{\braket{\psi|L_l^\dag L_l|\psi}}} \equiv L_l\ket{\psi}$. Perform measurement $M_m$ on $ L_l \ket{\psi} \to \frac{M_m L_l \ket{\psi}} {\sqrt{\braket{\psi|M^\dag_mM_m|\psi}}} \equiv M_m L_l \ket{\psi} \equiv N_{lm}\ket{\psi}$ as desired.
\end{tcolorbox}
(2.58) Suppose we prepare a quantum system in an eigenstate $\ket{\psi}$ of some observable $M$, with corresponding eigenvalue $m$. What is the average observed value of $M$ and the standard deviation?
\begin{tcolorbox}\textbf{Solution:}
	$E[M] = \braket{\psi|M|\psi} = m \braket{\psi|\psi} = m $, since $M\ket{\psi}= m \ket{\psi}$.
	Using the identity that $\Delta^2_M = E[M^2] - E[M]^2 = \braket{\psi|M^2|\psi} - m^2 = m^2 \braket{\psi|\psi}  - m^2 = 0$.
\end{tcolorbox}
(2.59) Suppose we have qubit in state $\ket{0}$ and we measure the observable $X$. What is the average value of $X$? What is the standard deviation of $X$?
\begin{tcolorbox}\textbf{Solution:}
	$$E[X] = \braket{0|X|0} = 0$$
	Since the Pauli matrices are unitary, $X^2 = I$
	$$E[X^2] = \braket{0|X^2|0} = \braket{0|I|0} = 1$$
	$$\iff \Delta_X = 1$$
\end{tcolorbox}
(2.60) Show that $v \cdot \sigma$ has eigenvalues $\pm 1$, and that the projectors onto the corresponding eigenspaces are given by $P_{\pm} = \frac{(I \pm v \cdot \sigma)}{2}$
\begin{tcolorbox}\textbf{Solution:}
	We convert into matrix form to calculate the eigenvalues.
	$$v\cdot \sigma = v_1 \sigma_1 + v_2 \sigma_2 + v_3 \sigma_3 \begin{bmatrix} v_3 & v_1 + iv_2 \\ v_1 - iv_2 & -v_3\end{bmatrix}$$
	$$det(v \cdot \sigma - \lambda) = (\lambda^2 -|v_3|^2) -|v_1|^2 + |v_2|^2 = 0 \iff \lambda^2 = |v_1|^2 + |v_2|^2 + |v_3|^3 = 1 $$
	$$\iff \lambda = \pm 1.$$
	Then, we show the projectors are given by $P_\pm = \frac{I\pm v \cdot \sigma}{2}$. We see that using spectral decomposition that $ v\cdot \sigma =  P_+ - P_-$. Using the property that $P_+ + P_- = I$, $\iff P_+ = \frac{I+v \cdot \sigma}{2},P_- = \frac{I- v\cdot \sigma}{2}$.
\end{tcolorbox}

(2.61) Calculate the probability of obtaining the result $+1$ for measurement of $v \cdot \sigma$ given that the state prior to measurement is $\ket{0}$. What is the state of the system after measurement if $+1$ is obtained?
\begin{tcolorbox}\textbf{Solution:}
	$P(+1) = \braket{0|P_+|0} = \frac{1+v_3}{2}$
	Post measurement state:
	$\ket{0}' = \frac12((1+v_3)\ket0 + (v_1-iv_2)\ket1)/\sqrt{\frac{1+v+3}{2}}$
\end{tcolorbox}
\section{2.2.6 POVM Measurements}
(Nielsen 2.62) Show any measurement where the measurement operators and the POVM elements coincide is a projective measurement.
	\begin{tcolorbox}\textbf{Solution:}
		\begin{proof}
		Suppose the measurement operators and the POVM elements coincide.
		Then, for all $M_m$, $M_m = E_m$. Since $E_m = M^{\dag}_mM_m = M_m$, we have idempotence (prop 1). Then, since $\sum_m E_m = I \iff \sum_m M_m = I$ and therefore we have completeness.\\
		Then, consider $\sum_m M_m = I \implies M_n \sum_m M_m = M_n$
		$$ \implies M^2_n + \sum_{m \neq n} M_nM_m  = M_n \iff M^2_n + \sum_{m \neq n} M_nM_m  = M_n$$
		$$ \sum_{m \neq n} M_n M_m= 0 \iff \sum_{m\neq n} \braket{\psi | M_nM_m | \psi} = 0 $$
		Since 0 is the sum of all strictly positive operators, it must be that each $M_nM_m$ is zero. Thus we have $M_mM_n = \delta_mn M_m$. Therefore if measurement operators and POVM elements coincide, then it is a projective measurement.
		\end{proof}
	\end{tcolorbox}
(2.63) Suppose a measurement is described by measurement operators $\{M_m\}$. Show there exist unitary operators $U_m$ such that $M_m = U_m \sqrt{E_m}$ where $E_m$ is the POVM associated to the measurement.
	\begin{tcolorbox}\textbf{Solution:}
	\begin{proof}
	Suppose we are given measurement operator $M_m$. Then, using polar composition there exists unitary operator $U$ such that $M_m = U_m J_m$, where $J_m$ is a positive operator. A positive operator always has a positive square root $\iff \exists E_m s.t. J_m = \sqrt{E_m}$, satisfying first condition for POVM (positive).\\
		See that $M^\dag_m = \sqrt{E_m}^\dag U_m^\dag =  \sqrt{E_m} U_m^\dag $
		$$\implies M_m^\dag M_m = \sqrt{E_m}U^{\dag}_m U \sqrt{E_m} = {E_m}$$
		Since $\sum_m M^\dag_m M_M  = \sum_m E_m = I$, the second condition is met.
		Since $M^\dag_M M_M = E_M \iff p(m) =\braket{\psi|E_m|\psi}$,  satisfying the third condition.
		Therefore we see a there exists a unitary operator $U$ such that $M_m = U_m \sqrt{E_m}$ where $E_m$ is the POVM associated to the measurement.
	\end{proof}
	\end{tcolorbox}
(2.64) Suppose Bob is given a quantum state chosen from a set $\ket{\psi}, \dots, \ket{\psi_m}$ of linearly independent states. Construct a POVM $\{E_1, E_2, \dots E_{m+1}\}$ such that if outcome $E_i$ occurs, $1 \leq i \leq m$, then Bob must know with certainty that he was given the state $\ket{i}$.
\begin{tcolorbox}
	Each $E_i = \ket{\psi_i}\bra{\psi_i}$ for $1 \leq i \leq m$, where $E_{m+1} =  I - \sum_i E_i$. This will yield the probability $1$ when applied to $\ket{\psi_i}$ and $0$ otherwise. We show this meets the criteria for POVM. We see that each $E_i$ is positive semi-definite, since as a projector its only eigenvalues are 0 and 1. We can also clearly see that adding all of the POVM operators equals the identity. 
	\end{tcolorbox}
\section{2.2.8 Composite Systems}
(2.66) Show the expected value of the observable $X_1Z_2$ for two qubit system measured in state $\frac{\ket{00} + \ket{11}}{\sqrt2}$
(2.68) Show the state $\ket{\psi} = \frac{1}{\sqrt{2}}\ket{00} + \frac{1}{\sqrt2}\ket{11}$ is entangled.
		\begin{tcolorbox}\textbf{Solution:}
		We show the state is entangled by showing that there is no possible $a_0,a_1,b_0,b_1$ such that $\ket{\psi} = (a_0\ket{0} + a_1\ket{1})(b_0\ket{0}+b_1\ket{1})$
		Suppose we can find valid $a_0,a_1,b_0,b_1$.\\
		Then, $\frac{1}{\sqrt2}\ket{00} + 1\sqrt{2}\ket{11} = a_0b_0\ket{00} + a_0b_1 \ket{01} + a_1b_0\ket{10} + a_1b_1 \ket{11}$
		Since $\ket{00},...,\ket{11}$ form a basis for $H^{\otimes2}$, this writing for $\ket{\psi}$ is unique $\iff \frac{1}{\sqrt{2}} = a_0b_0 = a_1b_1$ and $a_0b_1 = a_1b_0 = 0$.
		Then, we see that:
		$$a_0b_0 \times a_1b_1 = a_0 a_1 b_0 b_1=\frac12 \neq a_0b_1 \times a_1b_0 = 0$$
		Therefore there are no valid $a_0,a_1,b_0,b_1$.
\end{tcolorbox}
\section{4.2 Single qubit operations}
(4.1) Find the points on the Bloch sphere which correspond to the normalized eigenvectors of the different Pauli matrices.
\begin{tcolorbox}\textbf{Solution:}
	The eigenvectors of the Pauli matrices correspond to the axes of the Bloch sphere.
	For example, we solve for $\ket{\psi}_{x+} = \frac{1}{\sqrt2}(\ket0 + \ket1)$, which for $\ket{\psi} = \cos(\theta/2)\ket0 + e^{i \phi} \sin(\theta/2)\ket{1}$
$$\implies cos (\frac{\theta}{2}) = \frac{1}{\sqrt{2}} \implies \theta = \frac{\pi}{2}$$
$$e^{i\phi} \sin(\theta/2) = \frac{1}{\sqrt{2}} \implies \phi = 0$$
$$\implies v = (\cos 0 \sin \frac{\pi}{2}, \sin 0 \sin \frac{\pi}{2}, \cos{\frac{\pi}{2}}) = (1,0,0)$$
\end{tcolorbox}
(4.2) Let $x$ be a real number and $A$ a matrix such that $A^2 = I$. Show that $exp(iAx) = \cos(x)I + i \sin(x)A$
\begin{tcolorbox}\textbf{Solution:}
	$$exp(iAx) = 1 + iAx  + \frac{1}{2!}(iAx)^2 + \frac{1}{3!}(iAx)^3 + \frac{1}{4!}(iAx)^4 + \dots$$
	$$=  1 + iAx - \frac{1}{2!}x^2 - \frac{1}{3!}iAx^3 + \frac{1}{4!}x^4+\dots$$
	$$= 1 - \frac{1}{2!}x^2 + \frac{1}{4!}x^4 + \dots + iAx - \frac{1}{3!}iAx^3 + \dots$$
	$$= \cos(x)I + i\sin(x)A$$
\end{tcolorbox}

(4.3) Show that, up to a global phase, the $\frac{\pi}{8}$ gate satisfies $T = R_z(\frac{\pi}{4})$.
\begin{tcolorbox}\textbf{Solution:}
	\begin{align*}
		R_z(\frac{\pi}{4}) &= \begin{bmatrix}e^{-i\frac{\pi}{8}} & 0 \\ 0 & e^{i \frac{\pi}{8}}\end{bmatrix}\\
				  &\equiv e^{i\frac{\pi}{8}} \begin{bmatrix}e^{-i\frac{\pi}{8}} & 0 \\ 0 & e^{i \frac{\pi}{8}}\end{bmatrix}\\
				  &= \begin{bmatrix} 1 & 0 \\ 0 & e^{i \frac{\pi}{4}}\end{bmatrix} = T
\end{align*}
\end{tcolorbox}
(4.4) Express the Hadamard gate $H$ as a product of $R_x$ and $R_z$ rotations and $e^{i\phi}$ for some $\phi$.
\begin{tcolorbox}\textbf{Solution:}
	We see that $R_x(\frac{\pi}{2}]) = \frac{1}{\sqrt{2}}\begin{bmatrix} 1 & -i \\ -i & 1\end{bmatrix}$, which when multiplied on the left and right by $\begin{bmatrix} 1 & 0 \\ 0 & i\end{bmatrix}$ is the Hadamard gate. We then can see that $R_z(\frac{\pi}{2})  = \begin{bmatrix} e^{-i\frac{\pi}{4}} & 0 \\ 0 & e^{i\frac{\theta}{2}}\end{bmatrix} \implies e^{i\frac{\pi}{4}} R_z(\frac{\pi}{4}) = \begin{bmatrix} 1 & 0 \\ 0 & i\end{bmatrix} $  
	$$H = e^{i\frac{\pi}{2}}R_z(\frac{\pi}{2})R_x(\frac{\pi}{2}) R_z(\frac{\pi}{2})$$
\end{tcolorbox}
(4.5) Prove that $(\hat{n} \cdot \sigma)^2 = I$ and use this to verify Equation (4.8).
\begin{tcolorbox}\textbf{Solution:}
	$$(\hat{n} \cdot \sigma)^2 = (\begin{bmatrix}v_3 & v_1 - iv_2 \\ v_1 + iv_2 & -v_3 \end{bmatrix})^{2}$$
	$$= \begin{bmatrix} ||v||^2 & 0 \\ 0 & ||v||^2 \end{bmatrix}= I$$
	Then, (4.8) follows immediately from the exponential form of a complex number.
\end{tcolorbox}
(4.6) One reason why the $R_{\hat{n}} (\theta)$ operators are referred to as rotation operators is the following fact. Suppose a single qubit has a state represented by Bloch vector $\vec{\lambda}$, Then the effect of $R_{\hat{n}}(\theta)$ on the state is to rotate it by an angle $\theta$ about the $\hat{n}$ axis of the Bloch sphere. 
\begin{tcolorbox}
	Suppose we have $\vec{\lambda} = (\cos\varphi \sin \theta, \sin \varphi \sin \theta, \cos \theta) \equiv$
\end{tcolorbox}
\section{4.3 Controlled Operations}
\section{4.4 Measurement}
\section{4.5 Universal Quantum Gates}
\section{4.6 Simulation of Quantum Systems}
\end{document}
